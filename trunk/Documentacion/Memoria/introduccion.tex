\chapter{Introducción}
\label{chap:introducción}

``CITA''
``Estar preparado para la guerra es uno de los medios más eficaces para conservar la paz'' - George Washington

Hacer una breve descripción sobre el mundo de la seguridad informática, su repercusión en la sociedad y el futuro que tendrá en la revolución ``digital'' de todos los procesos de la vida cotidiana y empresarial con casos cómo el internet de las cosas y demás guisados.

También explicar un poco por encima del cometido a grandes rasgos por el que esa necesidad de vigilar dónde van los datos, conexiones y demás dentro de una máquina (O.S.) obliga a éste tipo de software (iptables, snort, glances) a funcionar y cómo una recopilación y procesamiento de los datos que estos generan permiten enfocar unas soluciones más prácticas y concisas sobre cómo actuar ante fuga de información (citar casos cómo wikileaks o robo de información en empresas por decir algo vamos)

\section{Objetivos}

El objetivo principal del proyecto es controlar los tipos de conexiones entrante y salientes de una máquina en sus diferentes protocolos de comunicación y mecanismos de gestión (firewall, ids, watchdogs, etc).

La aplicación deberá cumplir los siguientes requisitos:

\begin{itemize}
\item Ser una herramienta multiplataforma y que permita a cualquier usuario definir sus propias interfaces de gestión de incidencias (en el sentido de que pueda hacerse una clase.py en el kernel de la aplicación y esta sea capaz de hacer el resto simplemente con instanciar a dicha nueva clase.
\item Dotar de funcionalidad gráfica que permita extraer información en tiempo real con gráficas o mecanismos visuales (en web) del sistema de base de datos que ha procesado los inputs de las fuentes para las que ha sido configurada.
\item Dotar de una api interna que nos permita extraer información en tiempo real en un formato uniforme para la web o para que alguien pueda usar la funcionalidad del proyecto para su propio beneficio usando herramientas generadas en el back-end para otro tipo de aplicaciones (fumada guapa)
\end{itemize}

\section{Contexto: Historia sobre los sistemas de detección}
Contar alguna de las historias del libro el arte de la intrusión a modo de referencia, o buscar por ahí alguna relacionada con fuga de datos o similares.

\section{Alcance}

Que resultado se obtendrá con el proyecto y cómo se divide este internamente. Aquí podría poner el gráfico del controller, bd, visualizations, manager y demás.

Si se usa alguna notación también podría especificar cuál y porque.

\section{Visión global}

Cómo se divide internamente la memoria del pfc y que se va a hacer referencia en cada parte de la misma.

