\chapter{Introducción}
\label{chap:introducción}

\emph{Estar preparado para la guerra\\ es uno de los medios más eficaces\\ para conservar la paz\\ George Washington}\\

En el siglo XXI todo pasa por ser digital y si no, ya es que lo era antes de llegar a este punto. Quizás muchas de las tecnologías que hoy conocemos se basen en un sistema informatizado, ya sea en su formato de software o sistema embebido. Y es que todo pasa por ser una herramienta software diseñada para un propósito en concreto: un mecanismo de apertura de puertas mediante tarjetas RFID, procedimientos industriales o aplicados a alguna infraestructura crítica o de bien común, una aplicación del tiempo en tú terminal móvil, el propio sistema operativo con el que se puede éste el documento, etc. \\

Hay un sinfín de aplicaciones software que hacen nuestro día a día más llevadero y más fácil. Pero hay un punto que no todos conocen y es la necesidad de saber como se ha creado ése producto o como funciona realmente por su interior. En ése interior, a veces, podemos encontrar cosas que no estaban predestinadas a tener ese comportamiento y debido a ése comportamiento anómalo o imprevisto se generan situaciones de incertidumbre en las que el ser humano debe estar capacitado para afrontarlas. Dichas situaciones se suelen conocer con el término anglosajón de ``bug'' y sobre los bugs hay una especial categoría que se denominan fallas de seguridad o critical/several bugs. \\

Estas situaciones no previstas provocan que nuestro sistema, sea el que fuese, actúe de forma inesperada ante un input de información permitido o legítimo permitiendo un uso inadecuado de los recursos a los que se acceden mediante la aplicación. De este concepto o problema, surge en gran medida, el término de seguridad informática el cuál intenta abarcar y dar solución a estos problemas que pueden ir desde un simple fallo de desarrollo, a un fallo crítico que comprometa la seguridad o confidencialidad de los documentos de una empresa o gobierno. \\

Debido a esta problemática, surge la necesidad de analizar, monitorizar y generar sistemas de seguridad perimetral que permita a las empresas ver que tipo de tráfico interno se genera, que tipo de tráfico externo tiene y cómo se hace uso de él (navegación hacia el exterior, tuneles VPN, conexiones remotas a dispositivos, etc). Así pues, se podría decir que para obtener este tipo de eventos sobre protocolos, tráfico, DNS, IPs, VPNs,.. se tienen que configurar dispositivos de seguridad para la recolección de estos tipos de inputs o fuentes. \\

Una de las fuentes más conocidas dentro del mundo de la informática es el firewall, pero no así de las de un uso más extendido dentro del mundo doméstico sino el comercial o corporativo. Y dentro de los muchos tipos de software enfocados a tráfico (firewall) se encuentra el paquete de las distribuciones GNU/Linux: Iptables (software fruto del proyecto Netfilter para el kernel de GNU/Linux). Con esta herramienta se pueden definir políticas de filtrado de tráfico para cualquier tipo de protocolo TCP/UDP que queramos limitar entre el exterior y nuestra máquina y viceversa. Además, estas políticas nos permiten derivar dicho tráfico a archivos que podemos manipular obteniendo así los eventos que representan al tráfico generado por una máquina conectada a una red, que posteriormente podemos manipular para generar estadísticas o tipos de uso para una red. \\

\section{Contexto: Seguridad informática}

La seguridad informática es el proceso de mantener un aceptable nivel de percepción frente a un riesgo. Así pues ninguna organización se puede considerar ``segura'' en cualquier momento, más allá de la última comprobación que se realizo dentro de su política de seguridad.\\

Un proceso seguro se encuadra dentro de las siguientes 4 etapas: Evaluación, Prevención, Detección y Respuesta:

\begin{itemize}
\item Evaluación: es la preparación para las otras 3 etapas. Se considera cómo una acción separada porque se relaciona con las políticas, procedimientos, leyes, reglamentos, presupuestos y otras funciones de gestión, además de la propia evaluación técnica enfocada a la seguridad. No tener en consideración estos supuestos anteriores, podría dañar las etapas del diseño.
\item Prevención: es la aplicación de contramedidas para reducir la probabilidad de tener una situación comprometida.
\item Detección: es el proceso de identificación de intrusiones. Una intrusión se puede considerar como una violación de una política de seguridad o como un incidente de seguridad a nivel de software/dispositivo.
\item Respuesta: es el proceso de validar los inputs recogidos por la detección para tomar medidas que solucionen las intrusiones. El primer enfoque que debemos realizar consiste en restaurar la funcionalidad dañada y seguir recopilando información para tener claras las evidencias del atacante sobre nuestro sistema y poder así emprender las acciones legales que correspondan.
\end{itemize}

De estas etapas anteriores haremos incapié en la etapa de detección que es en la que se enmarca el ámbito de este proyecto. Dado que el principal objetivo es facilitar la monitorización de la información de seguridad.

\section{Motivación y objetivos del proyecto}

El objetivo principal del proyecto es desarrollar un software que permita recopilar y visualizar la información generada por las aplicaciones de monitorización y control de seguridad que se ejecutan en una máquina.\\

La motivación del mismo surge fruto de la necesidad de monitorizar un red corporativa a través de un mecanismo de gestión automatizada de eventos. Los pasos para la realización de este sistema se han modularizado y dividido en diferentes etapas que se acometarán como un todo dentro del proyecto de investigación VERITAS (\url{http://nesg.ugr.es/veritas/}) del Network Engineering \& Security Group (NESG - \url{http://nesg.ugr.es/}) perteneciente al área de Ingeniería Telemática de la Universidad de Granada.\\

\section{Estructura de la memoria}

En cuanto a la estructura de esta memoria del proyecto final de carrera, tras éste capítulo dónde se presentan los objetivos y la visión en general del proyecto, se expone el estado del arte y el análisis de requisitos previos al desarrollo software.\\

En el capítulo siguiente, veremos la etapa del diseño de software así cómo posterior evaluación del mismo.\\

Finalmente, se presentan las conclusiones generales obtenidas una vez realizado el proyecto, así también la planificación del mismo y estimación de costes.\\

Además, se presentan las referencias bibliográficas dónde se incluyen las fuentes consultadas para la elaboración de este proyecto, un resumen que engloba las generalidades fundamentales de la aplicación, una guía de utilización (manual de usuario), una guía de instalación, un compendio del software utilizado para el desarrollo y otro de los lenguajes de programación, y finalmente, la licencia completa del documento.\\
