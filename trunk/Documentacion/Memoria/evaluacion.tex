\chapter{Evaluación}
\label{chap:evaluación}

A continuación se dejan algunos ejemplos de las pruebas funcionales. Estos son test unitarios que se han realizado sobre la parte más importante del proyecto que es la base de datos y la integridad de dichos datos a la hora de poder operar con ellos desde la api/interfaz web.\\

Para una mayor profundidad en su estudio, estos se encuentran en el archivo ``tests.py'' dentro de la jerarquía del código fuente y también en el Apéndice 2: Tests [\ref{chap:tests}]. Pasemos a analizar cada uno de los test realizados sobre el modelo PacketEventsInformation de la base de datos.\\

\section{Test: PacketEventsInformation}

Clase del modelo relacional ORM - PacketEventsInformation. Es la clase que almacena todas las referencias al resto de clases internas de la arquitectura del sistema de base de datos.\\

\lstinputlisting{trozos-codigo/codigo-9-class.py}

\subsection{Método: setUp}
Con éste método se crean las instancias temporales en la base de datos \textbf{test} a la hora de la ejecución de la misma para comprobar su integridad. Para este caso tenemos que crear instancias paras todas las referencias que contiene un objeto de la clase PacketEventsInformation.

\lstinputlisting{trozos-codigo/codigo-9-set-up.py}

\begin{itemize}
\item ip-source: Instancia Clase Ips.
\item ip-dest: Instancia Clase Ips.
\item port-source: Instancia Clase Ports.
\item port-dest: Instancia Clase Ports.
\item mac-source: Instancia Clase Macs.
\item mac-dest: Instancia Clase Macs.
\item log-sources: Instancia Clase LogSources.
\item event: Instancia Clase Events.
\item PacketEventsInformation: Instancia a su propia clase.
\end{itemize}

Los métodos siguientes sirven para la consecución del test a la clase. \\

\subsection{Método: test\_id\_ip\_source}
Descripción: Comprobación de que el objeto de ip origen (referencia al modelo relacional) coincide con su asociado.\\
Pasos:
\begin{itemize}
\item Se consulta sobre la instancia Ips creada anteriormente, en el método setUp, si existe algún campo (\textbf{Ip}) cuyo valor coincida con el valor de: ``127.0.0.2''.
\item Se consulta sobre la instancia PacketEventsInformation creada anteriomente, si existe una coincidencia para el campo identificador \textbf{ID\_IP\_Source} con una referencia de una instancia similar, en este caso: \textbf{ip\_source}.
\item Usando el método interno de la clase TestCase, \emph{assertEqual}, se comprueba la integridad de la información que deben contener tanto la consulta a la instancia temporal de Ips como la consulta de la referencia interna asociada dentro de PacketEventsInformation.
\end{itemize}

\lstinputlisting{trozos-codigo/codigo-9-1.py}

\subsection{Método: test\_id\_ip\_dest}

Descripción: Comprobación de que el objeto de ip destino (referencia al modelo relacional) coincide con su asociado.\\
Pasos:
\begin{itemize}
\item Se consulta sobre la instancia Ips creada anteriormente, en el método setUp, si existe algún campo (\textbf{Ip}) cuyo valor coincida con el valor de: ``127.0.0.3''.
\item Se consulta sobre la instancia PacketEventsInformation creada anteriomente, si existe una coincidencia para el campo identificador \textbf{ID\_IP\_Dest} con una referencia de una instancia similar, en este caso: \textbf{ip\_dest}.
\item Usando el método interno de la clase TestCase, \emph{assertEqual}, se comprueba la integridad de la información que deben contener tanto la consulta a la instancia temporal de Ips como la consulta de la referencia interna asociada dentro de PacketEventsInformation.
\end{itemize}

\lstinputlisting{trozos-codigo/codigo-9-2.py}

\subsection{Método: test\_id\_source\_port}

Descripción: Comprobación de que el objeto de puerto origen (referencia al modelo relacional) coincide con su asociado.\\
Pasos:
\begin{itemize}
\item Se consulta sobre la instancia Ports creada anteriormente, en el método setUp, si existe algún campo (\textbf{Tag}) cuyo valor coincida con el valor de: ``ftp''.
\item Se consulta sobre la instancia PacketEventsInformation creada anteriomente, si existe una coincidencia para el campo identificador \textbf{ID\_Source\_Port} con una referencia de una instancia similar, en este caso: \textbf{port\_source}.
\item Usando el método interno de la clase TestCase, \emph{assertEqual}, se comprueba la integridad de la información que deben contener tanto la consulta a la instancia temporal de Ports como la consulta de la referencia interna asociada dentro de PacketEventsInformation.
\end{itemize}

\lstinputlisting{trozos-codigo/codigo-9-3.py}

\subsection{Método: test\_id\_dest\_port}

Descripción: Comprobación de que el objeto de puerto destino (referencia al modelo relacional) coincide con su asociado.\\
Pasos:
\begin{itemize}
\item Se consulta sobre la instancia Ports creada anteriormente, en el método setUp, si existe algún campo (\textbf{Tag}) cuyo valor coincida con el valor de: ``ssh''.
\item Se consulta sobre la instancia PacketEventsInformation creada anteriomente, si existe una coincidencia para el campo identificador \textbf{ID\_Dest\_Port} con una referencia de una instancia similar, en este caso: \textbf{port\_dest}.
\item Usando el método interno de la clase TestCase, \emph{assertEqual}, se comprueba la integridad de la información que deben contener tanto la consulta a la instancia temporal de Ports como la consulta de la referencia interna asociada dentro de PacketEventsInformation.
\end{itemize}

\lstinputlisting{trozos-codigo/codigo-9-4.py}

\subsection{Método: test\_protocol}

Descripción: Comprobación de que el protocolo del paquete coincide con su asociado.\\
Pasos:
\begin{itemize}
\item Se consulta sobre la instancia PacketEventsInformation creada anteriormente, en el método setUp, si existe algún campo (\textbf{Protocol}) cuyo valor coincida con el valor de: ``ICMP''.
\item Usando el método interno de la clase TestCase, \emph{assertEqual}, se comprueba la integridad de la información comparando que la devolución del método interno de la clase PacketEventsInformation es igual al argumento pasado al método: ``ICMP''.
\end{itemize}

\lstinputlisting{trozos-codigo/codigo-9-5.py}

\subsection{Método: test\_id\_source\_mac}

Descripción: Comprobación de que el objeto de mac origen (referencia al modelo relacional) coincide con su asociado.\\
Pasos:
\begin{itemize}
\item Se consulta sobre la instancia Macs creada anteriormente, en el método setUp, si existe algún campo (\textbf{TAG}) cuyo valor coincida con el valor de: ``Mac local1''.
\item Se consulta sobre la instancia PacketEventsInformation creada anteriomente, si existe una coincidencia para el campo identificador \textbf{ID\_Source\_MAC} con una referencia de una instancia similar, en este caso: \textbf{mac\_source}.
\item Usando el método interno de la clase TestCase, \emph{assertEqual}, se comprueba la integridad de la información que deben contener tanto la consulta a la instancia temporal de Macs como la consulta de la referencia interna asociada dentro de PacketEventsInformation.
\end{itemize}

\lstinputlisting{trozos-codigo/codigo-9-6.py}

\subsection{Método: test\_id\_dest\_mac}

Descripción: Comprobación de que el objeto de mac destino (referencia al modelo relacional) coincide con su asociado.\\
Pasos:
\begin{itemize}
\item Se consulta sobre la instancia Macs creada anteriormente, en el método setUp, si existe algún campo (\textbf{TAG}) cuyo valor coincida con el valor de: ``Mac local2''.
\item Se consulta sobre la instancia PacketEventsInformation creada anteriomente, si existe una coincidencia para el campo identificador \textbf{ID\_Dest\_MAC} con una referencia de una instancia similar, en este caso: \textbf{mac\_dest}.
\item Usando el método interno de la clase TestCase, \emph{assertEqual}, se comprueba la integridad de la información que deben contener tanto la consulta a la instancia temporal de Macs como la consulta de la referencia interna asociada dentro de PacketEventsInformation.
\end{itemize}

\lstinputlisting{trozos-codigo/codigo-9-7.py}

\subsection{Método: test\_raw\_info}

Descripción: Comprobación de que el log en formato RAW coincide con su asociado.\\
Pasos:
\begin{itemize}
\item Se consulta sobre la instancia PacketEventsInformation creada anteriormente, en el método setUp, si existe algún campo (\textbf{RAW\_Info}) cuyo valor coincida con el valor de: ``LOG RAW INFO''.
\item Usando el método interno de la clase TestCase, \emph{assertEqual}, se comprueba la integridad de la información comparando que la devolución del método interno de la clase PacketEventsInformation es igual al argumento pasado al método ``LOG RAW INFO''.
\end{itemize}

\lstinputlisting{trozos-codigo/codigo-9-8.py}

\subsection{Método: test\_tag}

Descripción: Comprobación de que la etiqueta especificada para el paquete coincide con su asociado.\\
Pasos:
\begin{itemize}
\item Se consulta sobre la instancia PacketEventsInformation creada anteriormente, en el método setUp, si existe algún campo (\textbf{Tag}) cuyo valor coincida con el valor de: ``Connection ICMP''.
\item Usando el método interno de la clase TestCase, \emph{assertEqual}, se comprueba la integridad de la información comparando que la devolución del método interno de la clase PacketEventsInformation es igual al argumento pasado al método ``Connection ICMP''.
\end{itemize}

\lstinputlisting{trozos-codigo/codigo-9-9.py}

\subsection{Método: test\_id}

Descripción: Comprobación de que el objeto de ip origen (referencia al modelo relacional) coincide con su asociado.\\
Pasos:
\begin{itemize}
\item Se consulta sobre la instancia Events creada anteriormente, en el método setUp, si existe algún campo (\textbf{Timestamp}) cuyo valor coincida con el valor de: Timestamp del sistema en la hora actual que se almacenó cómo atributo interno de la clase Test.
\item Se pasa como parámetro la referencia de la instancia anterior a la instancia temporal de la clase PacketEventsInformation (la destinataria del test).
\item Usando el método interno de la clase TestCase, \emph{assertEqual}, se comprueba la integridad de la información que deben contener tanto la consulta a la instancia temporal de Events como la devolución de la referencia interna asociada dentro de PacketEventsInformation.
\end{itemize}

\lstinputlisting{trozos-codigo/codigo-9-10.py}
