\chapter[Requisitos]{Especificación y análisis de requisitos}
\label{chap:requisitos}

Los requisitos funcionales y especificación de los mismos son los que a continuación se decriben.

Actualmente existen muchas tecnologías de monitorización de redes que a su vez tiene formato hardware, cómo controladores de sesiones de navegación de usuarios, firewall, proxys que enrutan el tráfico de una compañía hacia las redes externas, autenticación en sistemas VPN para acceso seguro sobre máquinas remotas, etc. El principal problema de estas tecnologías es que cada una pertenece a un ámbito distinto dentro del mismo departamento, es decir, pertenecen a las redes y a tráfico o flujo de comunicaciones pero se enmarcan en diferentes aspectos de cada uno de estos. \\

Para el caso que nos concierte en el proyecto, dentro del marco de investigación que define la totalidad de la infraestructura, la funcionalidad principal del mismo será:

\begin{itemize}
\item Definir los pasos para obtener recolección de logs de una fuente de seguridad. Configuración de rsyslog, logrotate y supervisord (este último en el caso de que sea necesario).
\item Una vez configurada la máquina, configurar la instalación para cada fuente en particular. El ámbito del proyecto se enfoca sobre iptables.
\item Realizar un sistema de parseo de logs para extraer la información necesaria para cada evento que se registre en el sistema.
\item Extracción de características de dichos eventos y procesamiento mediante un algoritmo de frecuencias para detectar las anomalías más comunes.
\item Sistema de gestión de base de datos en dónde se encuentren los datos en crudo recogidos, los procesados y los dispuestos para su visualización.
\item Panel de control dónde visualizar la información de ese nodo con total detalle de la información.
\item Compresión y envio de los datos a la capa superior dónde se alojará el SIEM.
\end{itemize}

\section[Configuración local]{Configuración local para procesamiento}
Los primeros pasos la la obtención de logs o eventos generados por la fuente de seguridad, iptables, serán los de configurar al sistema interno de correlación de logs rsyslog junto con el sistema de rotacion de logs logrotate. Para el caso de rsyslog tenemos que definir un filtro para que cualquier evento que genere el sistema con un determinado mensaje definido en las reglas de iptables, sea capturado y almacenado en un determinado directorio. También hemos dotado a los logs del sistema de timestamp con mayor precisión para poder diferenciar eventos con mayor afinamiento y cambiado la tupla de permisos a la hora de crear un archivo con \textbf{FileCreateMode}.

\begin{figure}[H]
  \begin{lstlisting}[language=bash]
    #
    # Use traditional timestamp format.
    # To enable high precision timestamps, comment out the following line.
    #
    #$ActionFileDefaultTemplate RSYSLOG_TraditionalFileFormat

    #
    # Set the default permissions for all log files.
    #
    $FileOwner root
    $FileGroup adm
    $FileCreateMode 0644
    $DirCreateMode 0755
    $Umask 0022

    # IPTABLES

    :msg,contains,"IPTMSG= " -/var/log/iptables.log
    :msg,regex,"^\[ *[0-9]*\.[0-9]*\] IPTMSG= " -/var/log/iptables.log
    :msg,contains,"IPTMSG= " ~

  \end{lstlisting}
  \caption{Configuración de iptables para Rsyslog}
\end{figure}
\pagebreak
También hemos de configurar Logrotate (más información sobre los campos visitar sección \ref{subsection:logrotate}):

\begin{figure}[H]
\begin{lstlisting}[language=bash]
/var/log/iptables.log
        {
                rotate 7
                daily
                missingok
                notifempty
                delaycompress
                compress
                postrotate
                        invoke-rc.d rsyslog restart > /dev/null
                endscript
        }
\end{lstlisting}
\caption{Configuración de iptables para Logrotate}
\end{figure}

Además tenemos que definir una regla específica para el demonio de rsyslog en la cual se filtre también por los campos que queramos de iptables:

\begin{figure}[H]
\begin{lstlisting}[language=bash]
# into separate file and stop their further processing
if  ($syslogfacility-text == 'kern') and \\
($msg contains 'IPTMSG=' and $msg contains 'IN=') \\
then    -/var/log/iptables.log
    &   ~

\end{lstlisting}
\caption{Configuración de iptables.conf para Rsyslog.d}
\end{figure}

Estos tres paso o configuraciones nos permiten redireccionar un log de iptables al directorio \textbf{/var/log/} y en conreto para el archivo \textbf{iptables.log}. Esta configuración entrará en efecto una vez hayamos reiniciado el servico rsyslog o en el próximo inicio de sesión sobre la máquina.

\pagebreak
\section[Logs]{Recogida y almacenamiento de logs}
Para el caso que nos ocupa, iptables tiene una forma de definir reglas internas para el filtrado de paquetes según el tipo de comunicación que se establezca contra la máquina o desde la máquina hacia el exterior.

\begin{figure}[H]
\begin{lstlisting}[language=bash]
iptables -A INPUT -p tcp -m tcp --dport 22 -j LOG --log-prefix "IPTMSG=Connection SSH "
\end{lstlisting}
\caption{Ejemplo de regla iptables}
\end{figure}

En la siguiente sección explicaremos en profundidad cada una de las opciones de la regla, pero a simple vista podemos observar que el mensaje asociado a la regla coincide con la palabra clave del filtro empleado en rsyslog: \textbf{IPTMSG=}. Así pues, una vez dicho evento se genere el sistema y syslog lo procese como un mensaje de un servicio determinado, en este caso iptables, rsyslog filtrará dicho mensaje según su configuración para almacenarlo posteriormente en \textbf{/var/log/iptables.log}.

\section{Iptables}

El servicio de firewall del kernel de GNU Linux, iptables, nos proporciona una interfaz de reglas y tablas en dónde podemos definir patrones o reglas que actuen sobre el tráfico que llega o sale desde nuestra máquina. Las reglas que se han definido por defecto son las siguientes (si queremos más filtros hay que implicar al protocolo y su puerto asociado con un mensaje):

\begin{figure}[H]
\begin{lstlisting}[language=bash]
# Generated by iptables-save v1.4.21 on Mon Jan 25 20:37:18 2016
*filter
:INPUT ACCEPT [0:0]
:FORWARD ACCEPT [0:0]
:OUTPUT ACCEPT [0:0]
-A INPUT -d 127.0.0.1/32 -p icmp -m icmp --icmp-type 8 -m state --state NEW,RELATED,ESTABLISHED -j LOG --log-prefix "IPTMSG=Connection ICMP "
-A INPUT -d 127.0.0.1/32 -p icmp -m icmp --icmp-type 8 -m state --state NEW,RELATED,ESTABLISHED -j DROP
-A INPUT -p tcp -m tcp --dport 22 -j LOG --log-prefix "IPTMSG=Connection SSH "
-A INPUT -p tcp -m tcp --dport 22 -j DROP
COMMIT
\end{lstlisting}
\caption{Configuracion reglas iptables}
\end{figure}

A continuación una breve explicación de cada opción o comando:
\begin{itemize}
\item -A: Añadir una nueva regla a una cadena de la tabla.
\item -d: Especificación para la ip destino, en este caso localhost con una máscara de subred determinada.
\item --dport: Especificación para el puerto destino al que se realizará la posible conexión o envío de paquetes.
\item -p: Especificación del protocolo del paquete.
\item -m: Especificación de matching, en este caso icmp o tcp dentro de la descripción del paquete.
\item --icmp-type: Extensión del tipo de ping que se va a procesar desde la regla.
\item --state: Tipo de paquete según conexión:
  \begin{itemize}
  \item NEW: Paquete que crea una nueva conexión.
  \item RELATED: Paquete que está relacionado a una conexión existente, pero que no es parte de ella, cómo un error ICMP o, un paquete que establece una conexión de datos FTP.
  \item ESTABLISHED: Paquete que pertenece a una conexión existente (que tuvo paquetes de respuesta).
  \item INVALID (no usado): Paquete que no pudo ser identificado por alguna razón: quedarse sin memoria o errores ICMP que no corresponden a ninguna conexión conocida. Normalmente estos paquetes deben ser descartados.
  \end{itemize}
\item -j: Acción de salto cuando encuentre dicha regla de paquetes. Se especifican dos acciones:
  \begin{itemize}
  \item LOG --log-prefix ``message'' : Cuando se encuentre dicha regla se recolecta cómo log de la misma adjuntando un mensaje para diferenciarla del resto de reglas.
  \item DROP : Cuando se encuentre dicha regla se descarta el paquete dentro del propio sistema. Al ir seguido LOG de un DROP el paquete se muestra en el registro de syslog para posteriormente ser eliminado del registro de almacenamiento de paquetes (Para esto usamos rsyslog que se encarga de almacenar dicho mensaje en un archivo de log).
  \end{itemize}
\end{itemize}

Así pues una vez tengamos un evento o paquete o log de iptables en nuestro sistema generados mediante ssh 127.0.0.1 o ping 127.0.0.1 obtendremos lo siguiente:

\begin{figure}[H]
\begin{lstlisting}[language=bash, breaklines=true]
[dom jul  3 17:04:03 2016] IPTMSG=Connection SSH IN=lo OUT= MAC=00:00:00:00:00:00:00:00:00:00:00:00:08:00 SRC=127.0.0.1 DST=127.0.0.1 LEN=60 TOS=0x00 PREC=0x00 TTL=64 ID=39454 DF PROTO=TCP SPT=47706 DPT=22 WINDOW=43690 RES=0x00 SYN URGP=0
[dom jul  3 17:04:05 2016] IPTMSG=Connection SSH IN=lo OUT= MAC=00:00:00:00:00:00:00:00:00:00:00:00:08:00 SRC=127.0.0.1 DST=127.0.0.1 LEN=60 TOS=0x00 PREC=0x00 TTL=64 ID=39455 DF PROTO=TCP SPT=47706 DPT=22 WINDOW=43690 RES=0x00 SYN URGP=0
\end{lstlisting}
\caption{Evento de ssh localhost en el sistema}
\end{figure}

Lo anterior correspondía a la salida del comando \textbf{\$ dmesg -T} que muestra todos los mensajes del sistema que se han pasado al syslog. La opción -T se usa para especificar el timestamp de cada mensaje con una mayor precisión.\\

\begin{figure}[H]
\begin{lstlisting}[language=bash, breaklines=true]
2016-07-03T17:03:35.664324+02:00 debian kernel: [23337.363387] IPTMSG=Connection SSH IN=lo OUT= MAC=00:00:00:00:00:00:00:00:00:00:00:00:08:00 SRC=127.0.0.1 DST=127.0.0.1 LEN=60 TOS=0x00 PREC=0x00 TTL=64 ID=39454 DF PROTO=TCP SPT=47706 DPT=22 WINDOW=43690 RES=0x00 SYN URGP=0
2016-07-03T17:03:37.668326+02:00 debian kernel: [23339.369692] IPTMSG=Connection SSH IN=lo OUT= MAC=00:00:00:00:00:00:00:00:00:00:00:00:08:00 SRC=127.0.0.1 DST=127.0.0.1 LEN=60 TOS=0x00 PREC=0x00 TTL=64 ID=39455 DF PROTO=TCP SPT=47706 DPT=22 WINDOW=43690 RES=0x00 SYN URGP=0
\end{lstlisting}
\caption{Log capturado y almacenado por rsyslog en /var/log/iptables.log}
\end{figure}

Lo anterior corresponde con la manipulación por parte de rsyslog del mensaje obtenido en syslog. Cómo podemos observar se añade un campo de timestamp de mayor precisión según la configuración que hemos establecido en rsyslog.conf para poder diferenciar con mayor exactitud eventos entre diferentes franjas de tiempo. \\

\begin{figure}[H]
\begin{lstlisting}[language=bash, breaklines=true]
++++++++++++++++++++++++++++++++++++++++++++++++++
--------------------------------------------------

 Procesando linea --> 2016-07-03T17:12:53.632264+02:00 debian kernel: [23896.003739] IPTMSG=Connection ICMP IN=lo OUT= MAC=00:00:00:00:00:00:00:00:00:00:00:00:08:00 SRC=127.0.0.1 DST=127.0.0.1 LEN=84 TOS=0x00 PREC=0x00 TTL=64 ID=64157 DF PROTO=ICMP TYPE=8 CODE=0 ID=14177 SEQ=7

--------------------------------------------------
++++++++++++++++++++++++++++++++++++++++++++++++++
---> Insertado registro: {'TAG': 'Connection ICMP', 'ID_Source_PORT': None, 'Protocol': u'ICMP', 'RAW_Info': '2016-07-03T17:12:53.632264+02:00 debian kernel 23896.003739 IPTMSG=Connection ICMP IN=lo OUT MAC=00:00:00:00:00:00:00:00:00:00:00:00:08:00 SRC=127.0.0.1 DST=127.0.0.1 LEN=84 TOS=0x00 PREC=0x00 TTL=64 ID=64157 DF PROTO=ICMP TYPE=8 CODE=0 ID=14177 SEQ=7 ', 'ID_Source_MAC': <Macs: 00:00:00:00:00:00:00:00:00:00:00:00:08:00>, 'ID_Source_IP': <Ips: 127.0.0.1>, 'ID_Dest_IP': <Ips: 127.0.0.1>, 'ID_Dest_PORT': None, 'ID_Dest_MAC': <Macs: 00:00:00:00:00:00:00:00:00:00:00:00:08:00>}

++++++++++++++++++++++++++++++++++++++++++++++++++
---> Fin de procesado de linea
\end{lstlisting}
\caption{Procesamiento del log capturado y almacenado en la bd interna de la aplicación}
\end{figure}
\pagebreak

\section[Parser]{Transformación de log en información útil: Parser}

Una vez hemos descrito los pasos a seguir para obtener un log para un evento iptables, llega la hora de procesar y hacer útil todos esos datos que tenemos. Para esta finalidad tenemos que usar expresiones regulares para generar un parser con el que ser capaz de traducir todos esos datos en información útil para nuestra aplicación.\\

\begin{figure}[H]
\lstinputlisting{trozos-codigo/codigo-4.py}
\caption{Instancia de la clase Pygtail y lectura de las líneas del log}
\end{figure}

Hacemos uso del módulo o paquete \href{https://pypi.python.org/pypi/pygtail}{Pygtail} que nos permite leer archivos de log que aún no han sido leídos. Es una especie de \textbf{\$ tail -f} a un archivo en concreto, pero usa un concepto de offset e inode para saber la última actualización y posición del archivo antes y después de ser abierto para así saber que parte de la última lectura se quedo en ejecución. Éste es otro punto importante dado que para hacer uso de esta funcionalidad debemos crear un archivo con el nombre del log, véase \textbf{/var/log/iptables.log} cuya extensión final sea offset (\textbf{/var/log/iptables.log.offset}) y sus permisos los siguientes:

\begin{figure}[H]
  \begin{lstlisting}[language=bash, breaklines=true]
    -rw-r--r-- 1 root adm 10391 jul  3 17:12 /var/log/iptables.log
    -rw-r--rw- 1 root root 14 jul  3 17:13 /var/log/iptables.log.offset
  \end{lstlisting}
  \caption{Permisos de los archivos iptables.log e iptables.log.offset}
\end{figure}

Una vez tengamos el archivo abierto y con sus líneas procesadas por Pygtail, es el momento de usar regex sobre los objetos string de cada línea de log. Para ello hacemos uso del método split para dividir por palabras, es decir, posiciones separadas en una lista a todas las coincidencias con una palabra que pudiera tener toda la cadena.\\

\begin{figure}[H]
\lstinputlisting{trozos-codigo/codigo-5.py}
\caption{Uso del método split sobre la entrada de lineas de log}
\end{figure}

Una vez separado por palabras toda la línea de log, pasamos a diferenciar entre las etiquetas que iptables pone a cada campo con su valor, es decir, una tupla key=>value. \\

\begin{figure}[H]
\lstinputlisting{trozos-codigo/codigo-6.py}
\caption{Obtenemos la tupla key=>value para cada etiqueta del log}
\end{figure}

Ya tenemos todas las etiquetas de los campos del log y ahora nos toca extraer su valor asociado para asignarlo al ORM de la base de datos, es decir, almacenar la información en la BD.\\

\begin{figure}[H]
\lstinputlisting{trozos-codigo/codigo-7.py}
\caption{Vamos asignando cada etiqueta y su valor a su asociado del ORM}
\end{figure}

Los siguientes pasos de comprobación de integridad de valores y demás se relegan a la visualización del interesado en el método process\_line del archivo código fuente iptables.py\\

\pagebreak
\section{Workflow}

\section{Extracción de características}

\section{Visualización de eventos}

\section{Compresión}
