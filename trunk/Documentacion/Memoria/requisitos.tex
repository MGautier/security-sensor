\chapter[Requisitos]{Especificación y análisis de requisitos}
\label{chap:requisitos}

Para el caso que nos concierte en el proyecto, dentro del marco de investigación que define la totalidad de la infraestructura, la funcionalidad principal del mismo será:

\begin{itemize}
\item Definir los pasos para obtener recolección de logs de una fuente de seguridad. Configuración de rsyslog, logrotate y supervisord (este último en el caso de que sea necesario).
\item Una vez configurada la máquina, configurar la instalación para cada fuente en particular. El ámbito del proyecto se enfoca sobre iptables.
\item Realizar un sistema de parseo de logs para extraer la información necesaria para cada evento que se registre en el sistema.
\item Sistema de gestión de base de datos en dónde se encuentren los datos en crudo recogidos, los procesados y los dispuestos para su visualización.
\item Panel de control dónde visualizar la información de ese nodo con total detalle de la información.
\end{itemize}

\section[Especificación]{Especificación de los requisitos}

En esta etapa del modelado de requisitos se captura el propósito general del sistema:

\begin{itemize}
\item Se analiza qué debe hacer el sistema.
\item Se obtiene una versión contextualizada del sistema.
\item Identifica y delimita el sistema.
\item Se determinan las características, cualidades y restricciones que debe satisfacer el sistema.
\end{itemize}

\subsection{Requisitos funcionales}

Los requisitos funcionales que se han obtenido en el sistema son los siguientes:

\begin{itemize}
\item Ser una herramienta multiplataforma y que permita a cualquier usuario definir sus propias interfaces de gestión de eventos.
\item Dotar de funcionalidad gráfica que permita extraer información en tiempo real con gráficas o mecanismos visuales (en web) del sistema de base de datos que ha procesado los inputs de las fuentes para las que ha sido configurada.
\item Dotar de una api interna que nos permita extraer información en tiempo real en un formato uniforme para la web o para que cualquier usuario pueda usar la funcionalidad del proyecto para su propio beneficio usando herramientas generadas en el back-end para otro tipo de aplicaciones.
\item Ser parte de un todo, en el que el todo sea un SIEM capaz de obtener información de las diferentes sondas o módulos, que en este caso, sería la solución desarrollada.
\item Desarrollar una sonda para procesar eventos logs de firewall.
\item Extracción y parseo de eventos de firewall Iptables.
\end{itemize}

\subsection{Requisitos no funcionales}

Los requisitos no funcionales son aquellos que describen cualidades o restricciones del sistema que no se relacionan de forma directa con el comportamiento funcional del mismo. A continuación se especifican los más importantes del sistema:
\begin{itemize}
\item No requiere un conocimiento específico del sistema una vez puesto en funcionamiento.
\item La aplicación tendrá manual de uso.
\item La base de datos estará implementada en un lenguaje objeto relacional como SQLite.
\item La aplicación estará realizada en el lenguaje de programación Python.
\item La interfaz debe reflejar claramente la distinción entre las distintas partes del sistema.
\item La documentación del código fuente será llevada a cabo mediante la aplicación Sphinx.
\item El sistema se desplegará sobre una versión GNU Linux Debian 8 Jessie.
\item El código fuente de la aplicación seguirá un estilo uniforme y normalizado para todos los módulos del mismo.
\item El formato de las fechas será dd/mm/yyyy a excepción de la utiliza para la funcionalidad api/events/day/<source>/yyyy-mm-dd.
\end{itemize}
