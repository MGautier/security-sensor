\chapter{Apéndice 1: Instalación}
\label{chap:instalacion}

\section{Instalación de Django y VirtualEnv}

Primero tenemos que instalarnos un entorno virtual de desarrollo para que nuestra aplicación no modifique nuestros paths internos de Python, o si bien queremos que todo lo que nos instalemos sea de uso general no tendríamos que hacer este paso. Para ello vamos a la página oficial del proyecto \href{https://virtualenv.pypa.io/en/stable/installation/}{virtualenv} y seguimos los puntos de instalación que nos indican.\\

Una vez configurado e instalado el paquete VirtualEnv en nuestra máquina, pasamos a utilizarlo dentro de nuestro proyecto base. Para ello simplemente, una vez clonado el mismo, vamos a la ruta ``trunk/version-1-0/webapp/'' y ejecutamos lo siguiente:\\

\begin{figure}[H]
  \begin{lstlisting}[language=bash]
    $ virtualenv .
    $ . bin/activate
  \end{lstlisting}
  \caption{Configuración de nuestro entorno virtual de Python}
\end{figure}

Una vez dispuesto nuestro entorno virtual vamos a instalar las dependencias necesarias para el funcionamiento de la aplicación. [Para este paso es necesario tener instalado el paquete \href{https://pip.pypa.io/en/stable/installing/}{pip}, que es el gestor de paquetes del lenguaje Python]\\

\begin{figure}[H]
  \begin{lstlisting}[language=bash]
    $ pip install -U pip
    $ pip install -r requirements.txt
  \end{lstlisting}
  \caption{Instalación de las dependencias del proyecto}
\end{figure}

Ahora es el momento de configurar nuestro proyecto Django para su ejecución:

\begin{figure}[H]
  \begin{lstlisting}[language=bash]
    $ cd secproject
    $ ./manage.py makemigrations
    $ ./manage.py migrate
    $ ./manage.py createsuperuser #Esto nos crea el superuser de administracion
  \end{lstlisting}
  \caption{Configuración de la base de datos y creación del super usuario}
\end{figure}

Si queremos especificar un usuario distinto a uno general para diferenciar entre entornos de desarrollo y producción, tendremos que acceder a la dirección ``http://127.0.0.1:8000/admin'', ingresar con el super usuario y definir nuevos usuarios para nuestra aplicación. Para nuestro propósito se ha definido un usuario llamado ``pfc'' para la parte de desarrollo de la aplicación. Una vez finalizada, se podrá establecer para un usuario normal o para otro específico.\\

\section{Rsyslog}

Ya tenemos configurado nuestro entorno de desarrollo de Django, ahora tenemos que configurar los servicios internos de la máquina para que correlen la información generada por iptables en nuestro caso. Así pues, vamos a configurar rsyslog para que redirija los eventos de iptables a ``/var/log/iptables.log''.\\

Incluímos las siguientes líneas en el archivo ``/etc/rsyslog.conf'':

\begin{figure}[H]
  \begin{lstlisting}[language=bash]
    # IPTABLES

    :msg,contains,"IPTMSG= " -/var/log/iptables.log
    :msg,regex,"^[ [0-9].[0-9]*] IPTMSG= " -/var/log/iptables.log
    :msg,contains,"IPTMSG= " ~
  \end{lstlisting}
  \caption{Configuración para filtrado de eventos de Iptables}
\end{figure}

Damos la tupla de permisos 0644 a la creación de archivos:

\begin{figure}[H]
  \begin{lstlisting}[language=bash]
   $ FileCreateMode 0644
  \end{lstlisting}
  \caption{Permisos a la creación de archivos}
\end{figure}

Para obtener timestamps más precisos tenemos que comentar la siguiente línea dentro del archivo de configuración de rsyslog:

\begin{figure}[H]
  \begin{lstlisting}[language=bash]
    #$ActionFileDefaultTemplate RSYSLOG_TraditionalFileFormat
  \end{lstlisting}
  \caption{Timestamp más preciso}
\end{figure}

\section{LogRotate}

Ahora tenemos que configurar el servicio LogRotate para que una vez finalizado el día, los archivos de logs antiguos, los comprima y los almacene en ``/var/log/''. Para ello vamos a la ruta ``/etc/logrotate.d/'' y creamos el archivo ``iptables'' con el siguiente contenido:

\begin{figure}[H]
  \begin{lstlisting}[language=bash]
    /var/log/iptables.log
    {
      rotate 7
      daily
      missingok
      notifempty
      delaycompress
      compress
      postrotate
      invoke-rc.d rsyslog restart > /dev/null
      endscript
    }
  \end{lstlisting}
  \caption{Configuración de LogRotate}
\end{figure}

\section{/var/log/iptables.log}

Tenemos que crear un archivo llamado ``iptables.log'' en ``/var/log'' con las siguientes características:

\begin{figure}[H]
  \begin{lstlisting}[language=bash]
    -rw-r--r-- 1 root adm 15604061 oct 26 20:28 /var/log/iptables.log
    $ chmod 644
    $ chown root:adm
  \end{lstlisting}
  \caption{Creación del archivo base de iptables.log}
\end{figure}

\section{Offset paquete PygTail}

El paquete PygTail usa para la lectura de logs dentro de nuestro sistema un archivo ``.offset'' del que consultará información relacionada del archivo del cual queremos obtener el texto correspondiente. Como por defecto, éste no puede ejecutarse con privilegios de super usuario dentro de la ruta ``/var/log'' o hacemos que nuestra aplicación corra directamente sobre super usuario (opción desaconsejada) o creamos el siguiente archivo para cada log que queramos procesar mediante PygTail.

\begin{figure}[H]
  \begin{lstlisting}[language=bash]
    -rw-r--rw- 1 root root    13 may  9 20:24 /var/log/iptables.log.offset
    $ chmod 646
    $ chown root:root
  \end{lstlisting}
  \caption{Creación del archivo offset de PygTail}
\end{figure}

\section{Rsyslog.d}

Tenemos que decirle al demonio de Rsyslog que todo lo que contenga el mensaje ``IPTMSG'' (que será nuestro mensaje prefijo para cada paquete obtenido mediante Iptables) lo mande a ``/var/log/itpables.log''. Para ello vamos a ``/etc/rsyslog.d/iptables.conf'' e introducimos lo siguiente:

\begin{figure}[H]
  \begin{lstlisting}[language=bash]
    # into separate file and stop their further processing
    if  ($syslogfacility-text == 'kern') and \\
    ($msg contains 'IPTMSG=' and $msg contains 'IN=') \\
    then    -/var/log/iptables.log
        &   ~
  \end{lstlisting}
  \caption{Creación del archivo de configuración iptables para el demonio Rsyslog}
\end{figure}

\section{Iptables}

Los pasos anteriores es para la recolección de eventos generados por Iptables dentro de nuestra máquina. Obviamente habrá que definir unas reglas de filtrado en Iptables cuyo campo de mensaje contenga la siguiente clave/prefijo ``IPTMSG= '' (incluir un espacio al final del mensaje). Aquí un ejemplo de las reglas que se han usado para la generación de eventos Iptables:

\begin{figure}[H]
  \begin{lstlisting}[language=bash]
    # Generated by iptables-save v1.4.21 on Mon Jan 25 20:37:18 2016
    *filter
    :INPUT ACCEPT [0:0]
    :FORWARD ACCEPT [0:0]
    :OUTPUT ACCEPT [0:0]
    -A INPUT -d 127.0.0.1/32 -p icmp -m icmp --icmp-type 8 -m state --state NEW,RELATED,ESTABLISHED -j LOG --log-prefix "IPTMSG=Connection ICMP "
    -A INPUT -d 127.0.0.1/32 -p icmp -m icmp --icmp-type 8 -m state --state NEW,RELATED,ESTABLISHED -j DROP
    -A INPUT -p tcp -m tcp --dport 22 -j LOG --log-prefix "IPTMSG=Connection SSH "
    -A INPUT -p tcp -m tcp --dport 22 -j DROP
    COMMIT
  \end{lstlisting}
  \caption{Reglas de Iptables usadas para el proyecto}
\end{figure}

\section{Web Server}

Ahora ya sólo nos queda configurar nuestro servidor web, que en este caso será Nginx. \textbf{Importante:} Previamente no debe haberse instalado una versión de servidor web Apache, sino habrá que desinstalar todo y aún así dará muchos problemas. Por lo que es altamente recomendable que la instalación este limpia del paquete apache en cualquiera de sus versiones.

\subsection{Nginx}

Instalación de Nginx:

\begin{figure}[H]
  \begin{lstlisting}[language=bash]
    $ sudo apt-get install nginx
  \end{lstlisting}
  \caption{Instalación del paquete Nginx}
\end{figure}

Configuración de Nginx:
\begin{itemize}
\item Vamos a la carpeta ``/etc/nginx/sites-available/'' y creamos nuestro archivo de configuración ``myproject.conf'' con el siguiente contenido:

  \begin{figure}[H]
    \begin{lstlisting}[language=bash]
      server {

        root /var/www/html;

        # Tipos de archivos index de nuestro sistema

        index index.html index.htm index.nginx-debian.html;

        # Nombre del servidor en local

        server_name localhost;

        location /static/ {
          alias <ruta-descarga-proyecto>/securityproject/trunk/ \\
                 version-1-0/webapp/secproject/secapp/static/;
          expires 30d;
        }

        location / {
          proxy_set_header X-Forwarded-For $proxy_add_x_forwarded_for;
          proxy_set_header Host $http_host;
          proxy_redirect off;
          proxy_pass http://127.0.0.1:8000;
          proxy_pass_header Server;
          proxy_set_header X-Real-IP $remote_addr;
          proxy_connect_timeout 10;
          proxy_read_timeout 10;


        }
      }
    \end{lstlisting}
    \caption{Configuración del servidor web en Nginx}
  \end{figure}
\item Una vez hemos escrito el archivo de configuración hacemos un enlace simbólico del mismo a otra carpeta de nginx, en este caso a ``sites-enabled/''
  \begin{figure}[H]
    \begin{lstlisting}[language=bash]
      $ sudo ln -s /etc/nginx/sites-available/myproject.conf /etc/nginx/sites-enabled/
    \end{lstlisting}
    \caption{Enlace simbólico a nuestra configuración previa}
  \end{figure}
\item Para comprobar que los archivos de configuración no tienen errores, ejecutamos el siguiente comando y si es éxito, reiniciamos el servicio:
  \begin{figure}[H]
    \begin{lstlisting}[language=bash]
      $ sudo nginx -t
      $ sudo service nginx restart
    \end{lstlisting}
    \caption{Comprobación de sintaxis de Nginx}
  \end{figure}
\item Ahora nos vamos al proyecto Django y lanzamos la instancia:
  \begin{figure}[H]
    \begin{lstlisting}[language=bash]
      $ ./manage.py runserver
    \end{lstlisting}
    \caption{Ejecución del servidor de Django}
  \end{figure}
\end{itemize}

Si la configuración se ha realizado correctamente, los contenidos estáticos de la web se verán en el navegador y no tendremos que entrar por el puerto 8000 sino por la dirección de loopback directamente: \href{http://127.0.0.1/secapp}{http://127.0.0.1/secapp} ó \href{http://127.0.0.1/admin}{http://127.0.0.1/admin}
