\chapter{Planificación y estimación de costes}
\label{chap:planificacion}

Esta claro que tendré que hacer un diagrama de gantt y demás. Aunque también puedo hacerme algún tipo de mini-aplicación que use la información de la api en taiga y luego la muestre temporalmente en formato gantt. Esto ya sería de curre extremo pero bueno, siempre existe la opción de hacer un gantt con los diferentes hitos/entregas que podrían ser perfectamente las reuniones que hemos tenido y demás.

Puedo establecer un periodo de unas 2 semanas por hito o entrega y definir que se ha hecho en esas etapas y demás cómo hice en la memoria (en plan muy desglosado) o hacer mención por encima con los commits que he subido al repo.


\section{Software utilizado}

El lenguaje usado es python 2.7, luego django con todo los paquetes asociados (poner algún enlace de referencia dónde se listen los paquetes más importantes que he tenido que instalar sin nombrar todas las dependencias de estas)

He usado cómo IDE Pycharm del projecto jetbrains

Para la memoria \LaTeX y para las gráficas tikz supongo.

\section{Licencia}

La licencia del proyecto para su posterior uso. En el actual repositorio de bitbucket no se ha especificado la licencia tal cuál, pero cuando ya este más estable el asunto también lo pondre en github para subirlo y demás. Ahí ya si que tendré que especificarla. En principio sino es para ningún propositio comercial, con la MIT sería suficiente. Sino alguna variante de la GPL o la Mozilla o similares.


