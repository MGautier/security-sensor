\chapter{Conclusiones}
\label{chap:conclusiones}

Durante la realización de éste proyecto, se han conseguido los siguientes resultados:

\begin{itemize}
\item Comprender el funcionamiento de los servicios del sistema: rsyslog, syslog, logrotate, nginx.
\item Comprender el funcionamiento del firewall del kernel, Iptables.
\item Desarrollar una solución software usando el lenguaje de programación Python.
\item Desarrollar una solución software, para sistemas web, usando el framework Django.
\item Manipulación de bases de datos relacionales usando un modelo orientado a objetos (ORM) proporcionado por el framework de desarrollo.
\item Diseñar el prototipo base para los sensores de recopilación de información de seguridad.
\item Diseñar una interfaz web donde el usuario pueda visualizar la información que el sensor ha analizado.
\item Despliegue de la aplicación en un servidor web en la nube para realizar demostraciones de la herramienta (Digital Ocean).
\item Documentar todo el proceso de realización del proyecto.
\end{itemize}
\pagebreak
A partir de los resultados obtenidos durante la realización de éste proyecto se han extraído las siguientes conclusiones:

\begin{itemize}
\item La aplicación \textit{Sensor para recopilación y visualización de información de seguridad en nodos de una red} es una herramienta que permite el análisis de información de dispositivos de seguridad para una máquina en una red.
\item El código fuente implementado, responde a los objetivos que debe cumplir el desarrollo del proyecto.
\item Con el diseño de la interfaz web se ha conseguido que un usuario sin conocimientos sobre el sistema, sea capaz de entender lo que se está registrando en él y que tipo de información de seguridad puede disponer a la hora de realizar un estudio de la misma.
\end{itemize}
