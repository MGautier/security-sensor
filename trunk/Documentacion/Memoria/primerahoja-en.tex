\documentclass{article}
\usepackage{estilosbase}
\begin{document}
\selectlanguage{english}
\pagenumbering{gobble}
\title{Sensor for collecting and displaying information security network nodes}
\author{Moisés Gautier Gómez}
\date{ }
\maketitle
\begin{center}
\small{\textbf{Keywords}}
\end{center}
\hspace*{.525in}{\small{Iptables, Computer security, Packets, Events, Python, Probe, Django, Web, Visualization}}

\begin{abstract}
  The main objective of the project is to develop a software to collect and display information generated by monitoring applications and security controls running on a machine.\\

  The motivation arises because of the need to monitor a corporate network through a mechanism of automated event management (SIEM). The steps for the implementation of this system are modularized and divided into different stages to be developed as a whole within the research project VERITAS (\url{http://nesg.ugr.es/veritas/}) of the Network Engineering \& Security Group (NESG - \url{http://nesg.ugr.es/}), that belogs to the area of Telematic Engineering of the University of Granada.\\

  For this purpose it is necessary to define the steps to obtain logs of a security source, configure the installation for that source, perform system parsing logs to extract the information, store the data in a persistent system (database) and visualize these data via a web interface.\\

  Finally, to test the efectiveness and analyze the performance of the software solution, a demonstration is done with actual live events processing from Iptables.\\
\end{abstract}
\end{document}
